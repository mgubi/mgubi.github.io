\documentclass{article}
\usepackage[english]{babel}
\usepackage{geometry,amsmath,amssymb,enumerate,theorem}
\geometry{a4paper, bottom=20mm, top=20mm}

%%%%%%%%%% Start TeXmacs macros
\newcommand{\TeXmacs}{T\kern-.1667em\lower.5ex\hbox{E}\kern-.125emX\kern-.1em\lower.5ex\hbox{\textsc{m\kern-.05ema\kern-.125emc\kern-.05ems}}}
\newcommand{\assign}{:=}
\newcommand{\mathd}{\mathrm{d}}
\newcommand{\tmtextbf}[1]{{\bfseries{#1}}}
\newenvironment{enumeratealpha}{\begin{enumerate}[a{\textup{)}}] }{\end{enumerate}}
{\theorembodyfont{\rmfamily\small}\newtheorem{exercise}{Exercise}}
%%%%%%%%%% End TeXmacs macros

\newcommand{\renderexercise}[2]{{\paddednormal{0.5fn}{0.5fn}{{\exercisename{#1{\exercisesep}}}#2}}}

\begin{document}

\setlength{\parindent}{0cm}{\small{\tmtextbf{Institute for Applied Mathematics
-- SS2020}}}\hspace{4cm}\hbox{}

{\small{Massimiliano Gubinelli}}

\


{\center{{\LARGE{\tmtextbf{V4F1 Stochastic Analysis -- Problem Sheet 10}}}

\ }}


{\scriptsize{Version 1, 2020.06.29. Tutorial classes: Mon July 6th 16--18 \
(Zoom) Min Liu \textbar  Wed July 8th 16--18 (Zoom) Daria Frolova.

Solutions in groups of 2 (at most). To be handled in {\LaTeX} or {\TeXmacs}
format via eCampus not later than 8pm Friday July 3rd. Use this sheet for your
solutions and write them under the corresponding exercise. Fill out your names
below.

\ }}

\tmtextbf{Names: XXXXXXXXXXXX/YYYYYYYYYYYYYY}

\

Let $(\Omega \assign C (\mathbb{R}_{\geqslant 0} ; \mathbb{R}), \mathcal{F},
\mathcal{F}_{\bullet}, \mathbb{P})$ the one dimensional Wiener space and $X$
the canonical process.

\hrulefill

\begin{exercise}
  [Pts 2+2+2+2+2] Find a predictable process $F$ such that
  \[ \Phi =\mathbb{E} [\Phi] + \int_0^{\infty} F_s \mathd X_s \text{} \]
  when $\Phi \in L^2 (\Omega, \mathcal{F}_T, \mathbb{P})$ is each of the
  following r.v. (with $T > 0$ fixed)
  \[ X_T^2, \quad e^{X_T}, \quad \int_0^T X_t \mathd t, \quad X_T^3, \quad
     \sin (X_T) . \]
  (One possible approach: for any $\Phi$ try to find a martingale $(M_t)_t$
  such that $M_T = \Phi$, and then apply Ito formula).
\end{exercise}

\hrulefill

\begin{exercise}[Pts 2+2+2]
  \label{ex:dens} We want to prove that the linear span of r.v. of
  the form
  \[ E (h) = \cos \left( \int h_s \mathd X_s \right) \exp \left( \frac{1}{2}
     \int h_s^2 \mathd s \right), \quad F (h) = \sin \left( \int h_s \mathd
     X_s \right) \exp \left( \frac{1}{2} \int h_s^2 \mathd s \right), \qquad h
     \in L^2 (\mathbb{R}_{\geqslant 0}), \]
  is dense in $L^2 (\Omega, \mathcal{F}, \mathbb{P})$ ($h$ is a deterministic
  function and the integrals are over $\mathbb{R}_{\geqslant 0}$).
  \begin{enumeratealpha}
    \item Show that if $G \in L^2 (\Omega, \mathcal{F}, \mathbb{P})$ is
    orthogonal to all $\{ E (h), F (h) : h \in L^2 (\mathbb{R}) \}$, then in
    particular
    \[ \mathbb{E} [G \exp (i \lambda_1 B_{t_1} + \cdots + i \lambda_n
       B_{t_n})] = 0 \]
    for all $\lambda_1, \ldots, \lambda_n \in \mathbb{R}$ and $t_1, \cdots,
    t_n \geqslant 0$.
    
    \item Deduce from this that $G$ is orthogonal to all functions of the from
    $\phi (B_{t_1}, \ldots, B_{t_n})$ with $\phi \in C_0^{\infty}$. [Hint: use
    Fourier transform]
    
    \item Conclude.
  \end{enumeratealpha}
\end{exercise}

\hrulefill

\begin{exercise}
  [Pts 4+4] Use the class of functions introduced in Exercise~\ref{ex:dens} to
  reprove the Brownian martingale representation theorem.
  \begin{enumeratealpha}
    \item Determine the martingale representation for functions $\Phi$ of the
    from
    \[ \Phi = \sum_i (a_i E (h_i) + b_i F (h_i)) \]
    where $a_i, b_i \in \mathbb{R}$, $h_i \in L^2 (\mathbb{R}_{\geqslant 0})$
    and the sum is finite.
    
    \item Use the density of such functions to approximate an arbitrary
    element $\Phi \in L^2$ and conclude.
  \end{enumeratealpha}
\end{exercise}

\

\hrulefill

\

\end{document}

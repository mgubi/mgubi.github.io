\documentclass{article}
\usepackage[english]{babel}
\usepackage{geometry,amssymb,enumerate,theorem,amsmath}
\geometry{a4paper, bottom=20mm, top=20mm}

%%%%%%%%%% Start TeXmacs macros
\newcommand{\TeXmacs}{T\kern-.1667em\lower.5ex\hbox{E}\kern-.125emX\kern-.1em\lower.5ex\hbox{\textsc{m\kern-.05ema\kern-.125emc\kern-.05ems}}}
\newcommand{\mathd}{\mathrm{d}}
\newcommand{\tmtextbf}[1]{{\bfseries{#1}}}
\newcommand{\tmtextit}[1]{{\itshape{#1}}}
\newenvironment{enumeratealpha}{\begin{enumerate}[a{\textup{)}}] }{\end{enumerate}}
{\theorembodyfont{\rmfamily\small}\newtheorem{exercise}{Exercise}}
%%%%%%%%%% End TeXmacs macros

\newcommand{\renderexercise}[2]{{\paddednormal{0.5fn}{0.5fn}{{\exercisename{#1{\exercisesep}}}#2}}}

\begin{document}

\setlength{\parindent}{0cm}{\small{\tmtextbf{Institute for Applied Mathematics
-- SS2020}}}\hspace{4cm}\hbox{}

{\small{Massimiliano Gubinelli}}

\


{\center{{\LARGE{\tmtextbf{V4F1 Stochastic Analysis -- Problem Sheet 6}}}

\ }}


{\scriptsize{Version 1, 2020.06.1. Tutorial classes: Mon June 8th 16--18 \
(Zoom) Min Liu \textbar  Wed June 10th 16--18 (Zoom) Daria Frolova.

Solutions in groups of 2 (at most). To be handled in {\LaTeX} or {\TeXmacs}
format via eCampus not later than \tmtextbf{8pm Friday June 5th}. Use this
sheet for your solutions and write them under the corresponding exercise. Fill
out your names below.

\ }}

\tmtextbf{Names: XXXXXXXXXXXX/YYYYYYYYYYYYYY}

\

\hrulefill

\begin{exercise}
  [Pts 2+2+2+2]\tmtextbf{(Brownian motion writes your name)} Prove that a
  Brownian motion in $\mathbb{R}^2$ will write your name (in cursive script,
  without dotted 'i's or crossed 't's). Let $B$ be a two dimensional Brownian
  motion on $[0, 1]$ and observe that $X^{(a, b)}_t = (b - a)^{1 / 2} (B_{a +
  (b - a) t} - B_a)$ \ for $t \in [0, 1]$ has the same law as $B$. Let $g :
  [0, 1] \rightarrow \mathbb{R}^2$ a smooth parametrization of your name. Let
  us agree that the Brownian motion $X^{(a, b)}$ spells your name (to
  precision $\varepsilon > 0$) if
  \begin{equation}
    \sup_{t \in (0, 1)} | X^{(a, b)}_t - g (t) | \leqslant \varepsilon .
    \label{eq:2}
  \end{equation}
  \begin{enumeratealpha}
    \item For $k \in \mathbb{N}$ let $A_k$ be the event that {\eqref{eq:2}}
    holds for $a = 2^{- k - 1}$ and $b = 2^{- k}$. Check that the events
    $(A_k)_{k \in \mathbb{N}}$ are independent and $\mathbb{P} (A_k)
    =\mathbb{P} (A_0)$ for all $k \geqslant 0$. Conclude that if $\mathbb{P}
    (A_0) > 0$ then \ infinitely many of the $A_k$s will occur almost surely.
    \
    
    \item Show that
    \begin{equation}
      \mathbb{P} [\sup_{t \in (0, 1)} | B_t | \leqslant \varepsilon] > 0.
      \label{eq:3}
    \end{equation}
    \item Using~{\eqref{eq:3}} and Girsanov's transform to show that
    $\mathbb{P} (A_0) > 0$ (Hint: construct a measure $\mathbb{Q}$ so that
    $B_t - g (t)$ is a Brownian motion)
    
    \item Prove that a similar result holds for $g$ only continuous.
  \end{enumeratealpha}
\end{exercise}

\hrulefill

\begin{exercise}
  [Pts 3] Let $(X, \mathbb{P})$ be a solution of the martingale problem with
  drift $b$ and diffusion $\sigma$. Generalise appropriately the Girsanov
  transform to construct a measure $\mathbb{Q}$ under which the process $X$
  solves a martingale problem with a different drift. For simplicity, assume
  that all the necessary integrability conditions are satisfied. (What takes
  the place of the Brownian motion?)
\end{exercise}

\hrulefill

\begin{exercise}
  [Pts 3+3+3] Given smooth, bounded functions $A : \mathbb{R}^d \rightarrow
  \mathbb{R}^d$, $V : \mathbb{R}^d \rightarrow \mathbb{R}$. Consider the
  operator $H (A)$ on $L^2 (\mathbb{R}^d)$ given by
  \[ H (A) = - \frac{1}{2} | \nabla - i A (x) |^2 + V (x) \]
  We will assume that this operator is self--adjoint (with suitable domain),
  bounded from below and with discrete spectrum. We will denote $E_0 (A)$ its
  smaller eigenvalue which we will assume simple (i.e. of multiplicity one).
  Let $\psi$ the complex valued solution to
  \[ \partial_t \psi (t, x) = - H (A) \psi (t, x), \qquad \psi_{} (0, x) =
     \psi_0 (x), \]
  which we will assume to exist, to be once differentiable in $t$ and twice in
  $x$ and be bounded with bounded derivatives.
  \begin{enumeratealpha}
    \item Find a suitable functions $B, C : \mathbb{R}^d \rightarrow
    \mathbb{C}$ with which we can give the following Feynman--Kac
    representation for $\psi$:
    \[ \psi (t, x) =\mathbb{E}_x \left\{ \psi_0 (X_t) \exp \left[ \int_0^t B
       (X_s) \mathd X_s + \int_0^t C (X_s) \mathd s \right] \right\} \]
    where under $\mathbb{E}_x$ the process $X$ is a $d$-dimensional Brownian
    motion starting at $x \in \mathbb{R}^d$.
    
    \item Prove that the lowest eigenvector of $H_A$ is strictly positive
    everywhere.
    
    \item Use the above representation to prove the \tmtextit{diamagnetic
    inequality}
    \[ E_0 (A) \geqslant E_0 (0) . \]
    {\small{\tmtextit{[Hint: take $\psi_0 (x) = 1$ and argue that $\psi (t, x)
    \simeq c e^{- E_0 t} \varphi (x) + o_t (1)$ where $H \varphi = E_0 (A)
    \varphi$ and conclude]}}}
  \end{enumeratealpha}
\end{exercise}

\hrulefill

\

\end{document}
